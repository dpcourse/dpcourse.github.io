\documentclass[11pt]{article}

% set this flag to 0 to remove comments
\def\comments{1}

% load the preamble
\usepackage{../preamble}

\begin{document}

%%%%%%%%%%%%%%%%%%%%%%%%%%%%%%%%%%%%
%%%%%%%%%%%%  Title  %%%%%%%%%%%%%%%
%%%%%%%%%%%%%%%%%%%%%%%%%%%%%%%%%%%%

\HWtitle{2}{Friday, March 12, 2021}

%%%%%%%%%%%%%%%%%%%%%%%%%%%%%%%%%%%%
%%%%%%%%%%  Beginning  %%%%%%%%%%%%%
%%%%%%%%%%%%%%%%%%%%%%%%%%%%%%%%%%%%

\noindent
\textbf{Collaboration and Honesty Policy Reminder:}
Collaboration in the form of discussion is allowed. However, all forms
of cheating (copying parts of a classmate’s assignment, plagiarism
from papers or old posted solutions) are NOT allowed. A rough rule of
thumb: you should be able to walk away from a discussion of a homework
problem with no notes at all and write your solution on your own.
%
Finding answers to problems on the
Web or from other outside sources (these include anyone not enrolled
in the class) is forbidden. 

\begin{compactitem}
\item {\em You must write up each problem solution by yourself without
assistance, even if you collaborate with others to solve the
problem.}

\item You must identify your collaborators. If you did not
work with anyone, you should write ``Collaborators: none."

\item Asking and answering questions in every forum the class provides (on Piazza, in class, and in
  office hours) is encouraged!

\item Even though look up answers is forbidden, using the web to find
  alternative explanations of concepts you need for the homework
  \emph{is} allowed, and encouraged. For example, you can look up
  background on probability and linear algebra, documentation for
  particular programming languages, etc.
    \end{compactitem}

 \paragraph{Problems to be handed in}
\begin{enumerate}[leftmargin=\parindent, itemsep=3ex]

\item     {\bf Medians.}
    Suppose we want to find the median of a list of real numbers
      $\dataset = (\data_1,...,\data_n)$ that lie in the set $\set{1,...,R}$. 

      Consider an instantiation of the exponential mechanism based on the
      following score function: 
      For every $y\in \set{1,...,R}$, let
        $$q(y; \dataset) = - \left|\sum_{i=1}^n \textit{sign}(y - \data_i)\, \right|
        \,$$
        where $$\textit{sign}(z)=
        \begin{cases}
          1 & \text{if }z>0 \,,\\
          0 &          \text{if }z=0 \,,\\
          -1 & \text{if }z<0 \,.\\
        \end{cases}$$ If all the input values are distinct, this score
        is 0 exactly when $y$ is a valid median for $\dataset$. In
        general, the score will be minimized at the true median.


        \begin{enumerate}
        \item Show that $q$ has sensitivity at most 1 when
          neighboring data sets are allowed to differ by the insertion
          or deletion of one entry.

          \item Let $A_\eps$ be the algorithm one gets by instantiating the
            exponential mechanism with score $q$, parameter $\eps$ and
            output set $\mathcal{Y} = \set{1,...,R}$. Show that there is a constant $c>0$ such that:
            for every data set $\bf x$, for
            every $R$ and $\epsilon<1$, and for every
            $\beta\in (0,1)$, the probability that $A_\eps(\dataset)$ samples a value
            $y$ with
            $|\textit{rank}_{\dataset}(y) -  n/2 | > c\cdot \frac {\ln(R) +
              \ln(1/\beta)}{\eps}$ is at most $\beta$.
            Here $rank_{\dataset}(y) \in\set{0,1,...,n}$ is the position $y$ would have in the
            sorted order of $\dataset$.

            For this part, it is ok to
            assume distinct data values, so that the rank of a value
            is uniquely defined.

            [\emph{Hint:} How does $\textit{rank}_{\dataset}(\cdot)$
            relate to $q(\cdot ; {\dataset})$?  Look at the ratio
            between the
            probability mass of a true median  and the probability mass of
            an element with very low or high rank.]

        \end{enumerate}

          \item \textbf{Implementing the Median Algorithm.} Implement the report-noisy-max version of this
            algorithm. Your code should take a set $\data$ of real
            values as input along with parameters $R$ and $\eps$. The
            first step should be to round all entries to the nearest
            integer in $\{1,...,R\}$. The output should be a single
            real number.

           Test your code on data drawn from the following
           distributions, using $\epsilon = 0.1$ and $n \in \{50,100,500, 2000, 10000\}$:
           \begin{itemize}
           \item Gaussian: $\cN(R/4, R^2/10)$ for $R \in
             \set{100,1000,10000}$.
           \item Poisson: $\textit{Poi}(50)$ for $R \in 
             \set{100,1000,10000}$.
           \item Bimodal: $R=1,000$; each data point uniform over two
             values $\set{\frac{R}{2}-k, \frac{R}{2}+k}$ for $k = 10, 100, 200$.
             \end{itemize}
             For each setting of parameters (data distribution, $R$
             and $n$), sample 50 data sets from the distribution, and
             run the algorithm 10 times on each. Collect the following
             statistics:
             \begin{itemize}
             \item Average error in rank: how far from $n/2$ is the
               rank of the output?
             \item Standard deviation of error in rank
             \item Standard deviation of error in rank \emph{among runs on
               the same data set.}
           \end{itemize}

         
        
\item \textbf{Report Noisy Max.} Prove that report noisy max with Exponential noise is differentially
  private (Exercise 2.1 from the notes for Lecture 6).

\item         \textbf{Histograms.}
  Consider the following algorithm for releasing histograms.
  
  \begin{algorithm}[H]
  	\DontPrintSemicolon
    \caption{Stable Histogam$(\dataset; \eps,\delta)$}
    \KwIn{$\dataset$ is a multi-set of values in $\univ$. }
    \For{every $z\in\univ$ that appears in $\dataset$}
    {
      $\tilde c_z = \#\set{i: x_i =z} + \Lap(1/\eps)$\;
    }
    Release the set of pairs $\set{(z, \tilde c_z) : \tilde c_z >
      \tau}$ where $\tau = 1+ \frac{\ln(1/\delta)}{\eps}$. 
  \end{algorithm}

  \begin{enumerate}
  \item Show that for any domain $\univ$, Algorithm 1 is
    $(\eps,\delta)$-differentially private when neighboring data sets
    are allowed to differ by the insertion or deletion of one value.

  
    \emph{Hint:} The delicate part of this result is that we add noise
    only to counts of non-empty bins. (For example, if we were
    counting how many people live on each square mile of land in
    Alaska, most of the bins would be empty, but others would have
    lots of people.) There are two kinds of adjacent data sets: those
    where the set of nonempty bins changes, and those where it does
    not.  You may need the following simple concentration bound for
    Laplace random variables: If $Y\sim\Lap(\lambda)$, then for every
    $t>0$, we have $\Pr( Y >\lambda t ) \leq \frac 1 2 \exp(-t)$.
  

    \item Prove that the Stable Histograms
    algorithm is not $(\eps',0)$ differentially private for any finite
    positive value $\eps'$. [\emph{Hint}: Give two neighboring data sets and
    a histogram $y$ such that $y$ is  a possible output for only one of
    the two data sets.]
  \end{enumerate}

\end{enumerate}


\end{document}
