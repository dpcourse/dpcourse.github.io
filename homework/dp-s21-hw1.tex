\documentclass[11pt]{article}

% set this flag to 0 to remove comments
\def\comments{1}

% load the preamble
\usepackage{../preamble}

\begin{document}

%%%%%%%%%%%%%%%%%%%%%%%%%%%%%%%%%%%%
%%%%%%%%%%%%  Title  %%%%%%%%%%%%%%%
%%%%%%%%%%%%%%%%%%%%%%%%%%%%%%%%%%%%

\HWtitle{1}{Friday, February 12, 2021}

%%%%%%%%%%%%%%%%%%%%%%%%%%%%%%%%%%%%
%%%%%%%%%%  Beginning  %%%%%%%%%%%%%
%%%%%%%%%%%%%%%%%%%%%%%%%%%%%%%%%%%%

\noindent
\textbf{Collaboration and Honesty Policy Reminder:}
Collaboration in the form of discussion is allowed. However, all forms
of cheating (copying parts of a classmate’s assignment, plagiarism
from papers or old posted solutions) are NOT allowed. A rough rule of
thumb: you should be able to walk away from a discussion of a homework
problem with no notes at all and write your solution on your own.
%
Finding answers to problems on the
Web or from other outside sources (these include anyone not enrolled
in the class) is forbidden. 

\begin{compactitem}
\item {\em You must write up each problem solution by yourself without
assistance, even if you collaborate with others to solve the
problem.}

\item You must identify your collaborators. If you did not
work with anyone, you should write ``Collaborators: none."

\item Asking and answering questions in every forum the class provides (on Piazza, in class, and in
  office hours) is encouraged!

\item Even though look up answers is forbidden, using the web to find
  alternative explanations of concepts you need for the homework
  \emph{is} allowed, and encouraged. For example, you can look up
  background on probability and linear algebra, documentation for
  particular programming languages, etc.
    \end{compactitem}

 \paragraph{Problems to be handed in}
\begin{enumerate}[leftmargin=\parindent, itemsep=3ex]
\item (Randomized
  response---Exercise 3 from Lecture 1's in-class exercises.) Consider
  the second randomized response mechanism described in Lecture 1: On
  input a function $\querypred:\univ\to\{0,1\}$ and a dataset $\dataset =
  (\data_1,...,\data_n)$ where each $\data_i$ is in domain $\univ$, compute
  $$Y_i =  \begin{cases}
    \querypred (x_i) & \text{w.p.  } \frac{e^\epsilon}{e^\epsilon+1}\, ,\\
    1-\querypred (x_i) & \text{w.p.  } \frac{1}{e^\epsilon+1} \, .\\
  \end{cases}$$
  and return $ (Y_1,...,Y_n)$.
  
  Give a procedure that, given the outputs $Y_1,...,Y_n$ from randomized response on
  input $\data_1,..., \data_n$, returns an estimate $A$ such that $$\sqrt{\ex{}{\paren{A -
        \sum_{i=1}^n \querypred (\data_i)}^2}} = \frac{e^{\epsilon/2}}{e^\epsilon -1} \sqrt{n}
\, .
$$
The randomness here is over the choices of the randomized response mechanism.
[\emph{Hint:} How would you choose $a,b \in \R$ so that $\ex{}{aY_i-b}
= x_i$?]


\item (Global Sensitivities.) For all of the following cases, assume
  we have a datset $\dataset = (\data_1,\dots,\data_n) \in \univ^n$ and
  a function $f \from \univ^n \to \R^d$ (for some $d$ that varies
  between the different parts of the question).  For each of the
  following functions $f$ and data domains $\univ$,
  (i) give as tight a bound as you can on the global sensitivity of
  the function $f$.  If the sensitivity is not bounded, answer
  $\infty$. (ii) Give as tight a bound as you can on the diameter of
  the range of $f$: specifically,
  what is the maximum difference (in $\ell_1$ norm) between two possible outputs of $f$? 

    \begin{enumerate}
    \item The high-dimensional mean $f(\dataset) = \frac 1 n \sum_{i=1}^n \data_i$ when 
      $\univ = \{v \in \R^d: \|v\|_1\leq 1\}$.
    
    \item The unnormalized covariance matrix when
      $\univ = \{v \in \R^d: \|v\|_1\leq 1\}$ . Here $f(x)=C$ is a
      $d\times d$ symmetric matrix with entries
      $C_{j,k}=\sum_{i=1}^n \data_i(j) \data_i(k)$, and $x_i(j)$
      refers to the $j$-th entry of vector $\data_i$. In other words, if we write
      the input as a $n\times d$ matrix $X$ whose rows are the $\data_i$,
      then $f(\dataset) = X^T X$. To measure sensitivity, we think of $f(\dataset)$
      as a single vector of length $d^2$. 

    \item The median $f(\dataset) = \mathrm{median}(\data_1,...,\data_n)$, when $\univ=[0,1]$.

     \item Consider an arbitrary domain $\univ$, and $k$ different
       predicates $f_1,..,f_k$ (so each $f_j$ maps $\univ$ to
       $\{0,1\}$). Let $f(x) = \max_{j=1,...,k} \frac 1 n \sum_{i=1}^n f_j(x_i)$.

     \item The US government uses a specific algorithm $H$ to decide,
       based on state populations, how many of the 435 seats in the
       House of Representatives will go to each state.\footnote{The
         exact algorithm doesn't matter here, but it's called the
         ``Huntingdon-Hill method''. } This
       ``apportionment'' is recalculated after each decennial
       Census. For simplicity, think of the Census data set as just
       recording each person's state of residence, so
       $\univ=\{1,...,50\}$.

       Let $f(\dataset)$ be the function that reports the number of people
       whose residence would need to change in order to change the
       apportionment $H(\dataset)$. In other words, $f(\dataset)$ is the Hamming
       distance to the nearest data set $\by$ with $H(\by) \neq H(\dataset)$. 
       
    \item Suppose we a fixed set of vertices $V$
      (independent of the data set). Our data set is a list of edges:
      each $x_i$ is a pair of vertices $(u,v)$ (so that $\univ=V\times
      V$). Let $G_\dataset$ be the resulting graph, and let $f(\dataset)$ denote the
      number of connected components in $G_\dataset$. 

    \item Let $\univ= V\times V$ as above, and let $f(\dataset)$ be the length
      of the shortest path between two specific vertices (let's say
      vertices number 1 and 2, assuming they are numbered 1 through
      $|V|$). (If no path exists, the length of the shortest path is
      $\infty$).

    \item  Let $\univ= V\times V$ as above, and let $f(\dataset)$ be the number
      of edge-disjoint paths from vertex 1 to vertex 2. 
      
    \end{enumerate}
  

  

\item (In-class Exercise 3 from Lecture 3: More accurate reconstruction with more random queries.)  In this question we'll explore how to interpolate between the two reconstruction theorems we've seen.  Specifically, we will prove a version of Theorem 2.5 that gives a more accurate reconstruction when we have $k \gg n$ queries.  Suppose we have the following version of Claim 2.6 from the lecture notes:
\begin{clm} 
	Let $t \in \{-1,0,+1\}^n$ be a vector with at least $m$ non-zero entries and let $u \in \zo^{n}$ be a uniformly random vector.  Then for every parameter $2 \leq w \ll  2^{m}$
	\begin{equation}
		\pr{}{| u \cdot t | \geq  \frac{\sqrt{m \log w}}{10}} \geq \frac{1}{w}
	\end{equation}
\end{clm}
Using this claim, prove the following theorem
\begin{thm}
	If we ask $n^2 \ll k \ll 2^n$ queries, and all queries have error at most $\alpha n$, then with extremely high probability, the reconstruction error is at most $O(\frac{\alpha^2 n^2}{\log(k/n)})$.
\end{thm}

How does this theorem compare to the reconstruction attacks we've seen for $k \approx n^2$?  What about $k \approx 2^{\sqrt{n}}$?  What about $k \approx 2^n$?
  

\item 
  A social media company wants
  to provide information to its advertisers on how often people click
  on their ads. You work for an advertiser. Each hour, the company
  gives you the demographic profile of one user who saw your ad. This
  demographic profile is rich enough for you to identify the user in
  your own database. They also give you a running count of the number
  of users who have actually clicked on your ad. They add a little bit
  of noise ``to protect users' privacy'', but you suspect that you can
  derive a good guess as to who, exactly clicked on your ads.

  We'll model this as follows: there is a sequence $\dataset$ of secret bits, $\data_1,
 \data_2, \data_3, ..., \data_n \in \{0,1\}$. The bit $\data_i$ indicates that
  the $i$th user clicked on the ad.

  The mechanism for releasing the running counter works as follows: 
  At each time step $i=1, 2, ...,
  n$, select a uniformly random bit $Z_i$ (independently of
  previous ones) and output
  $$a_i = \sum_{j=1}^i \data_j +
  Z_i\,.$$ (In other words, at each step, we flip a coin to decide
  whether we should add one fake person to the counter or not).

  How well can one recover the vector $\dataset$ from the sequence of
  outputs $\vec a $?  The answer depends on what we know about
  $\dataset$ ahead of time.
%
  We will consider two situations:

  \begin{enumerate}
  \item The bits of $\dataset$ are uniformly random and
    independent. The only input to the attacker is $\vec a$ (the vector of noisy counters). 
  \item The bits of $\dataset$ are uniformly random and independent, but
    the attacker has some extra information. For each $i$, the
    attacker has a guess $w_i$ which is equal to $\data_i$ with
    probability 2/3, independently for each $i$. The attacker's inputs
    consist of $\vec a$ and the vector of guesses $\vec w$. 
  \end{enumerate}

  For each of these situations, your job is to come up with as good an
  algorithm as you can to guess the entries of $\dataset$. You should
  evaluate your algorithm by coding it up (in a language of your
  choice) and plotting the fraction of bits of $\dataset$ recovered by
  your algorithm for $n= 100, 500, 1000, 5000$, and, if you can, $50,000$. (You
  should run the algorithm on fresh random inputs at least 20 times
  for each values of $n$, and report the mean and standard deviation
  for each one.) You'll need to code up the release mechanism, too, in
  order to run the experiments.

  You may use standard linear algebra libraries (such as those
  provided by \texttt{numpy} in Python). You may use standard solvers
  for linear systems or linear programs. 

 
  You should submit a single PDF file containing
  \begin{compactitem}
  \item A written description of your algorithms (whatever mix of
    English and pseudocode you like is fine---it should be clear,
    readable, and self-contained). (Maximum 1 page each)
  \item Plots of your results for each of the two settings
    \item Discussion of your results (Maximum 1/3
      page)
  \item Documented code
  \item (Optional) A link to a public repository with your code,
    ideally in a notebook format.
  \end{compactitem}

  To receive full credit, your algorithms should recover significantly
  more than $3/4$ of the bits of $\dataset$ in case 1, and
  significantly more than $4/5$ of the bits of $\dataset$ in case
  2. There is no specific running time requirement, but you should ideally be
  able to run experiments with $n=50,000$, so exponential time
  algorithms won't be feasible, and even quadratic-time algorithms
  will be painfully slow. You can use any of the techniques
  from class, or ones you invent. See how well you can do!




\end{enumerate}


\end{document}
