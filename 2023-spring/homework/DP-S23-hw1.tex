\documentclass[11pt]{article}

% set this flag to 0 to remove comments
\def\comments{1}

% load the preamble
\usepackage{../preamble23}

\begin{document}

%%%%%%%%%%%%%%%%%%%%%%%%%%%%%%%%%%%%
%%%%%%%%%%%%  Title  %%%%%%%%%%%%%%%
%%%%%%%%%%%%%%%%%%%%%%%%%%%%%%%%%%%%

\HWtitle{1}{Sunday, February 26, 2023}

%%%%%%%%%%%%%%%%%%%%%%%%%%%%%%%%%%%%
%%%%%%%%%%  Beginning  %%%%%%%%%%%%%
%%%%%%%%%%%%%%%%%%%%%%%%%%%%%%%%%%%%

\noindent
\textbf{Collaboration and Honesty Policy Reminder:}
Collaboration in the form of discussion is allowed. However, all forms
of cheating (copying parts of a classmate’s assignment, plagiarism
from papers or old posted solutions) are NOT allowed. A rough rule of
thumb: you should be able to walk away from a discussion of a homework
problem with no notes at all and write your solution on your own.
%
Finding answers to problems on the
Web or from other outside sources (these include anyone not enrolled
in the class) is forbidden. 

\begin{compactitem}
\item {\em You must write up each problem solution by yourself without
assistance, even if you collaborate with others to solve the
problem.}

\item You must identify your collaborators. If you did not
work with anyone, you should write ``Collaborators: none."

\item Asking and answering questions in every forum the class provides (on Piazza, in class, and in
  office hours) is encouraged!

\item Even though look up answers is forbidden, using the web to find
  alternative explanations of concepts you need for the homework
  \emph{is} allowed, and encouraged. For example, you can look up
  background on probability and linear algebra, documentation for
  particular programming languages, etc.
    \end{compactitem}

 \paragraph{Problems to be handed in}
\begin{enumerate}[leftmargin=\parindent, itemsep=3ex]

  
\item (\textbf{Node sensitivity}) Imagine we have a data set
  that comes from a simple social network with $n$ people. Each node in the graph is a
  person. For each person, we have the following data: a unique
  identifier $id$, their income $a_{id} \in [0,1]$
  and their friend list $F_{id}$, which is a list of identifiers
  of other nodes. $F_{id}$ can have any size---it can be empty, or
  consist of \emph{all} othe nodes, or anything in between. We
  assume that friendship is symmetric: if Alice is on Bob's friend
  list, then Bob is on Alice's.

  We'll say two graphs are neighbors if they differ by changing the
  data for one node, including $a_{id}$ and the list of edges connected
  to that node. Notice that changing one person's data can potentially
  affect everyone's else list of friends.
  Because of that, natural things
  we would like to compute can have high sensitivity. For example, the
  number of connected components in the graph can go from $n$ (the
  current number of nodes) to $1$
  by changing the friend list of  a single node. 
  
  \begin{enumerate}
  \item As a function of $n$, what is the global sensitivity of each
    of the following functions? Give the best upper bounds and lower
    bounds that you can. (It should be possible to given an exact
    answer, like $n^2-1$), for each of these.)

    How does the sensitivity compare to the range of the function
    (that is, the difference between the largest and smallest possible
    values it can return)?
    \begin{enumerate}
    \item the \emph{number of edges} in the graph
    \item the \emph{number of triangles} in the graph
    \item the \emph{diameter of the graph} (this is the largest
      possible length of the shortest path betweent two nodes)
    \item \emph{Distance from bipartiteness}: this is the smallest
      number of nodes that must be changed for the graph to be
      bipartite.
    \end{enumerate}
  \item \emph{Income correlation}: How correlated are friends'
    incomes? Let $\mu$ be the average income in the graph 
    $\mu = \frac 1 n \sum_{id} a_{id}$. The income correlation is
    $$g(x) =  \frac 1{2\# (edges)} \sum_{id} \sum_{id' \in F_{id}}
    (a_{id}-\mu)(a_{id'}-\mu)\, .$$
    (We multiply the number of edges by 2 since each edge gets
    counted twice in the formula.)
    
    This quantity ranges between 0 and 1. Design a differentially
    private algorithm which approximates $g$ with additive error $\pm O(1/n)$
    on all graphs for which $\# (edges) \geq n^2/20$. (The constant
    20 is arbitrary. In fact, one can get vanishing relative error
    under much weaker conditions.)
    \end{enumerate}
         
  

\item (In-class Exercise 3 from Lecture 3) \textbf{More accurate
    reconstruction with more random queries}

  In this question we'll explore how to interpolate between the two reconstruction theorems we've seen.  Specifically, we will prove a version of Theorem 2.5 that gives a more accurate reconstruction when we have $k \gg n$ queries.  Suppose we have the following version of Claim 2.6 from the lecture notes:
\begin{clm} 
	Let $t \in \{-1,0,+1\}^n$ be a vector with at least $m$ non-zero entries and let $u \in \zo^{n}$ be a uniformly random vector.  Then for every parameter $2 \leq w \ll  2^{m}$
	\begin{equation}
		\pr{}{| u \cdot t | \geq  \frac{\sqrt{m \log w}}{10}} \geq \frac{1}{w}
	\end{equation}
\end{clm}
Using this claim, prove the following theorem
\begin{thm}
	If we ask $n^2 \ll k \ll 2^n$ queries, and all queries have error at most $\alpha n$, then with extremely high probability, the reconstruction error is at most $O(\frac{\alpha^2 n^2}{\log(k/n)})$.
\end{thm}

How does this theorem compare to the reconstruction attacks we've seen for $k \approx n^2$?  What about $k \approx 2^{\sqrt{n}}$?  What about $k \approx 2^n$?



  \item (In-class exercise 6 from lecture 4) \textbf{Differential Privacy and
    Reconstruction Attacks}

  Suppose $A$ is an $\eps$-differentially private algorithm that takes input
  $\dataset = (\data_1,\data_2,...,\data_n) \in \bit{n}$ (so each
  person's secret information is just one bit). Consider an algorithm $B$
  that attempts to reconstruct the input from $A$'s output: on input
  $A(\dataset)$, it outputs a guess $\tilde\dataset$. Show that, for
  every algorithm $B$:
  if $\dataset$ is selected uniformly at random from $\bit n$, and the
  algorithm $B$ has access only to the output of $A$ (nothing else),
  then
  \[
    \ex{\substack{ \dataset \in_r \bit{n} \\ \tilde \dataset = B(A(\dataset))}}{
      \text{\# errors}\paren{\tilde \dataset,
        \dataset}} \geq  \dfrac{n}{e^\eps + 1}
  \]
  Here, $\text{\# errors}(y,x)$ denotes the number of positions in
  which two vectors disagree (also called the Hamming distance). \footnote{
  In other words: when $\eps$ is small, differentially private
  algorithms do not allow for non-trivial reconstruction
  attacks. Even with no output at all, an attacker can always guess
  about $\frac n 2$ of the bits
  of $\dataset$ in expectation (for example, by guessing the all-zeros
  string). The result above says that a attack based on differentially
  private output cannot do much better in expectation. }
  
  \emph{Hints:} Use linearity of expectation. The number of errors can
  be written as a sum of random variables $E_i$ (for $i=1$ to $n$),
  where
  $$E_i=
  \begin{cases}
    1 & \text{if }\tilde \dataset_i = \data_i \, ,\\
    0 & \text{otherwise.}
  \end{cases}
$$ What can you say about
  the conditional distribution of $\data_i$ given a particular output
  $A(\dataset) = a$? How big or small can $\Pr(\data_i = 1 |
  A(\dataset)=a)$ be? Given that, what is the largest possible
  probability that $E_i=1$? What does that tell you about $E_i$'s
  expected value? It might be helpful to think about what happens when
  $A$ is the randomized response mechanism, though your final proof
  should apply to any $\eps$-DP algorithm.

  

  
          \item \textbf{Implementing the Median Algorithm.} Implement
            the report-noisy-max version of the median algorithm from
            Lecture 6's In-class Exercises.

            Your code should take a set $\data$ of real
            values as input along with parameters $R$ and $\eps$. The
            first step should be to round all entries to the nearest
            integer in $\{1,...,R\}$. The output should be a single
            integer.

           Test your code on data drawn from the following
           distributions, using $\epsilon = 0.1$ and $n \in \{50,100,500, 2000, 10000\}$:
           \begin{itemize}
           \item Gaussian: $\cN(R/4, R^2/10)$ for $R \in
             \set{2^7,2^{10}, 2^{16}}$.
           \item Poisson: $\textit{Poi}(50)$ for $R \in 
             \set{2^7,2^{10}, 2^{16}}$. (NB: The distribution does not
             change as the range increases.)
           \item Bimodal: $R=2^{10}$; each data point uniform over two
             values $\set{\frac{R}{2}-k, \frac{R}{2}+k}$ for $k = 10, 100, 200$.
             \end{itemize}
             For each setting of parameters (data distribution, $R$
             and $n$), sample 50 data sets from the distribution, and
             run the algorithm 10 times on each. Collect the following
             statistics:
             \begin{itemize}
             \item Average error in rank: how far from $n/2$ is the
               rank of the output (on average over all 500 runs)?
             \item Standard deviation of error in rank (over all 500
               runs): how much does this error vary from run to run?
             \item Average over data sets of the standard deviation of error in rank \emph{among runs on
               that data set.} (This tells you whether different data
             sets from the same distribution have different
             distributions on the error.)
           \end{itemize}

           
             
             

\end{enumerate}


\end{document}
