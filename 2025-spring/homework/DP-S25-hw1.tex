\documentclass[11pt]{article}

% set this flag to 0 to remove comments
\def\comments{1}

% load the preamble
\usepackage{../preamble25}

\begin{document}

%%%%%%%%%%%%%%%%%%%%%%%%%%%%%%%%%%%%
%%%%%%%%%%%%  Title  %%%%%%%%%%%%%%%
%%%%%%%%%%%%%%%%%%%%%%%%%%%%%%%%%%%%

\HWtitle{1}{Sunday, February 16, 2025}

%%%%%%%%%%%%%%%%%%%%%%%%%%%%%%%%%%%%
%%%%%%%%%%  Beginning  %%%%%%%%%%%%%
%%%%%%%%%%%%%%%%%%%%%%%%%%%%%%%%%%%%

\noindent
\textbf{Collaboration and Honesty Policy Reminder:}
Collaboration in the form of discussion is allowed. However, all forms
of cheating (copying parts of a classmate’s assignment, plagiarism
from papers or old posted solutions) are NOT allowed. A rough rule of
thumb: you should be able to walk away from a discussion of a homework
problem with no notes at all and write your solution on your own.
%
Finding answers to problems on the
Web or from other outside sources (these include anyone not enrolled
in the class) is forbidden. 

\begin{compactitem}
\item {\em You must write up each problem solution by yourself without
assistance, even if you collaborate with others to solve the
problem.}

\item You must identify your collaborators. If you did not
work with anyone, you should write ``Collaborators: none."

\item Asking and answering questions in every forum the class provides (on Piazza, in class, and in
  office hours) is encouraged!

\item Even though looking up answers is forbidden, using the web, generative AI, 
  or similar resources to find
  alternative explanations of concepts you need for the homework
  \emph{is} allowed, and encouraged. Asking Deepseek to solve a problem for you is not ok; asking it for an explanation of Chernoff bounds or for examples of encoding quardatic constraints in Gurobi is fine. You must \textbf{document your use of outside sources} and describe it at a high level in your solutions.
    \end{compactitem}

 \paragraph{Problems to be handed in}
\begin{enumerate}[leftmargin=\parindent, itemsep=3ex]

  
  

\item (In-class Exercise 3 from Lecture 3) \textbf{More accurate
    reconstruction with more random queries}

  In this question we'll explore how to interpolate between the two reconstruction theorems we've seen.  Specifically, we will prove a version of Theorem 2.5 that gives a more accurate reconstruction when we have $k \gg n$ queries.  Suppose we have the following version of Claim 2.6 from the lecture notes:
\begin{clm} 
	Let $t \in \{-1,0,+1\}^n$ be a vector with at least $m$ non-zero entries and let $u \in \zo^{n}$ be a uniformly random vector.  Then for every parameter $2 \leq w \ll  2^{m}$
	\begin{equation}
		\pr{}{| u \cdot t | \geq  \frac{\sqrt{m \log w}}{10}} \geq \frac{1}{w}
	\end{equation}
\end{clm}
Using this claim, prove the following theorem
\begin{thm}
	If we ask $n^2 \ll k \ll 2^n$ queries, and all queries have error at most $\alpha n$, then with extremely high probability, the reconstruction error is at most $O(\frac{\alpha^2 n^2}{\log(k/n)})$.
\end{thm}

How does this theorem compare to the reconstruction attacks we've seen for $k \approx n^2$?  What about $k \approx 2^{\sqrt{n}}$?  What about $k \approx 2^n$?

\newpage
\item (\textbf{Node sensitivity}) Imagine we have a data set
  that comes from a simple social network with $n$ people. Each node in the graph is a
  person. For each person, we have the following data: a unique
  identifier $u$, their income $a_{u} \in [0,1]$
  and their friend list $F_{u}$, which is a list of identifiers
  of other nodes. $F_{u}$ can have any size---it can be empty, or
  consist of \emph{all} other nodes, or anything in between. We
  assume that friendship is symmetric: if Alice is on Bob's friend
  list, then Bob is on Alice's.

  We'll say two graphs are neighbors if they differ by changing the
  data for one node, including $a_{u}$ and the list of edges connected
  to that node. Notice that changing one person's data can potentially
  affect everyone's else list of friends.
  Because of that, natural things
  we would like to compute can have high sensitivity. For example, the
  number of connected components in the graph can go from $n$ (the
  current number of nodes) to $1$
  by changing the friend list of  a single node. 
  
  \begin{enumerate}
  \item As a function of $n$, what is the global sensitivity of each
    of the following functions? It should be possible to given an exact
    answer, like $n^2-1$, for each of these, but it is also ok to give upper and lower bounds that differ by a constant (e.g. ``$GS_f$ is at least $\binom n 2$ and at most $n^2$'').

    How does the sensitivity compare to the range of the function
    (that is, the difference between the largest and smallest possible
    values it can return)?
    \begin{enumerate}
    \item the \emph{number of edges} in the graph
    \item the \emph{number of triangles} in the graph
    \item the \emph{diameter of the graph} (this is the largest
      possible length of the shortest path betweent two nodes; we define the diameter of a disconnected graph to
      be the largest diamter of any of its connected components). 
    \item \emph{Distance from bipartiteness}: this is the smallest
      number of nodes that must be changed for the graph to be
      bipartite.

    \end{enumerate}
  \item \emph{Income correlation}: How correlated are friends'
    incomes? Let $\mu$ be the average income in the graph 
    $\mu = \frac 1 n \sum_{u} a_{u}$. The income correlation is
    $$g(x) =  \frac 1{2\# (edges)} \sum_{u} \sum_{w \in F_{u}}
    (a_{u}-\mu)(a_{w}-\mu)\, .$$
    (We multiply the number of edges by 2 since each edge gets
    counted twice in the formula.)
    
    This quantity ranges between 0 and 1. Design an $\eps$-differentially
    private algorithm which approximates $g$ with additive error $\pm O(1/\eps n)$
    on all graphs for which $\# (edges) \geq n^2/20$. (The constant
    20 is arbitrary. In fact, one can get vanishing relative error
    under much weaker conditions.)

    %\item Suppose we want to add less noise to the number of triangles. One way to do this is ...

    \end{enumerate}
         
\item \textbf{(Histograms with Randomized Response)} We saw a randomized response protocol $RR_\eps$ where each participant inputs a single bit and announces a single output bit. What if each person's input is one of $d$ different possibilities (say, the name of their favorite artist on Spotify) and we want to estimate a \emph{histogram} of how many people have each possible input? Can we do this so that each person can do the randomization of their data entirely on their own?

Consider the following approach, which we will call approach A: 
\begin{itemize}
  \item Each person encodes their input $x_i\in [d]$ as a ``one-hot'' indicator vector $e_{x_i} = (0,0,\dots , 0, 1, 0, \cdots, 0) \in \bit{d}$ where the one is in entry $x_i$. 
  \item Each person applies the randomizer $RR_{\eps/2}$ to each entry of this vector, to get a new vector $\vec Y_i \in \bit{d}$, which they announce publicly. 
\end{itemize}

Answer the following questions: 
\begin{enumerate}
  \item Prove that the approach A is $\eps$-differentially private. 
  \item Demonstrate that, given the $\vec Y_i$'s, one can estimate the counts of every item with expected error $O\paren{\frac{\sqrt{n\ln d}}{\eps}}$. You many do this \emph{either} by giving a proof, \emph{or} by coding up the mechanism and the estimation algorithm, and plotting the accuracy it achieves for different values of $n,d,\eps$. You should give three plots, each of which shows the effect of $n,d$, or $\eps$.
  \item (*, optional for glory) Now consider approach B, where each person $i$ does the following: 
  \begin{itemize}
    \item Announce a random vector $u_i\in\bit{d}$ (independent of their data)
    \item Announce $RR_\eps(u_i(x_i))$ where $u_i(x_i)$ denotes entry number $x_i$ of the vector $u_i$. 
  \end{itemize}
  Show that approach B has similary utility to approach A (with the right estimation procedure). Also, show that it is $\eps$-differentially private regardless of how $u_i$ is chosen. In particular, we can choose the $u_i$'s pseudorandomly and send only a seed used to generate $u_i$. This reduces the communication in the protocol dramatically! The pseudorandomness is only required for arguing accuracy.
\end{enumerate}





\item \textbf{(Reconstruction from 2-way Marginals)}
  In class, we considered settings where the data set is binary. But we know that complex machine learning models sometimes encode even very complex inputs. 
  Let's consider a setting where one can, somewhat surprisingly, reconstruct the entire data set from sufficiently rich statistics. 

  Suppose we have a data set consisting of $n$ rows (one per person), with $d$ binary attributes per row. A 1-way marginal describes the distribution in the data set of one particular attribute. Since the attributes are binary, a one-way marginal is the just count of how many 0's and 1's there are in a particular column. The ($d$-dimensional) vector of all one-way marginals is just the sum of the rows of the data. 

  \begin{center}
    \begin{tabular}{c||c|c|c|c|}
      & $a$ & $b$& $c$& $d$ \\
      \hline
      Alice & 0 & 1 & 1& 0\\
      Bob   & 1 & 0 & 1& 0\\
      Carol & 0 & 1 & 0 & 0\\    
      \hline
    \end{tabular}
    \hfill     
    \begin{tabular}{c||c|c|c|c|}
      & $a$ & $b$& $c$& $d$ \\
      \hline
      One-way  & 1 & 2 & 2& 0\\
      marginals & & & & \\
      (count of 1's) & & & & \\
    \end{tabular}
    \hfill
    \begin{tabular}{|c|c|}
      \multicolumn{2}{|c|}{Two-way marginal} \\
      \multicolumn{2}{|c|}{for $a,b$} \\
      \hline
      00 & 0\\
      01 & 2\\
      10 & 1 \\
      11 & 0\\
    \end{tabular}
  \end{center}
  

  A \emph{two-way marginal} describes the distribution of a pair of attributes. It consists of 4 counts, one for each of the pairs of bits 00, 01, 10, and 11. If we already know the one-way marginals for both attributes, we just need a single additional number such as the count of 11's. In the example above, if we reveal the one-way marginals together with the count of 11's for the attributes $a,b$ (which is 0), then we can conclude the that 01 appears twice, 10 appears once, and 00 appears 0 times. 

  One- and two-way marginals are basic summary statistics about a dataset. The question is, when are they sufficiently rich to pin down the data set exactly? 

  Mathematically, if $X \in \bit{0,1}^{n\times d}$ is the original sensitive table, then releasing the one- and two-way marginals is equivalent to releasing the matrix $X^T X$ and the parameters $n$ and $d$. (Why? Check this for yourself.)
  \begin{enumerate}
    \item \label{item:dimensioncount} How many different constraints does one learn by seeing $X^T X$? How does this compare to the number of variables in $X$? For which values of $n$ and $d$ would you expect to be able to recover $X$ from $X^T X$? 

    \item\label{item:pyprog} In Python, program the following experiment for values of $n$ and $d$ between 1 and 12. Specifcally, try all pairs $n,d$ where $n\in \set{3,6, 12}$ and $d \in \set{1,2,\dots,12}$. Repeat the following 20 times for each pair $(n,d)$: 
    \begin{enumerate}
      \item Choose $X$ uniformly at random and compute $M = X^T X$.
      \item Use Gurobi's free Python package API (see below) to find a binary matrix $\tilde X$  such that $\tilde X^T \tilde X = M$.
      \item Check if $\tilde X$ and $X$ are the same up to re-ordering of the rows. (How can you check this quickly?).
    \end{enumerate}
    For each setting of $n$ and $d$, record the fraction of times that the experiment led to exact reconstruction. 
    \item Do the results of Part \ref{item:pyprog} and \ref{item:dimensioncount} agree? Why or why not?
    \item (*, optional for glory) Gurobi's exact solver is too slow to solve this problem meaningfully for larger dimensions, at least on my laptop. Design a heuristic that scales to $d$ and $n$ in the hundreds, and carry out the experiment with $n=200$ and different settings of $d$. For what values of $d$ does your heuristic reconstruct the data exactly with reasonable probability (say at least 5\% of the time)?  
  \end{enumerate}
  
  \emph{What to submit:} You should submit a report answering the questions above, with your code attached as an appendix. (You have to submit a PDF on Gradescope, so you'll have to ``print'' your code to PDF and concatenate the two PDF files.)

  \emph{Gurobi:} To get Gurobi up and running, you will need to install it. This \href{https://support.gurobi.com/hc/en-us/articles/17278438215313-Tutorial-Getting-Started-with-the-Gurobi-Python-API}{web page} has instructions on getting Gurobi installed and up and running. I will also distribute an example Python notebook that might be helpful. 
  As for the optional problem... feel free to use any standard Python packages. Your code should runnable on a typical laptop or PC. (For $n=12$ and $d=12$, my use of Gurobi was taking under 5--10 minutes to run all 20 trials.)
\end{enumerate}


\end{document}
